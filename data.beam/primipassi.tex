\section{Primi passi}
\frame{\transfade\centering
\frametitle{Tipologie di documento}
\begin{itemize}
  \item<2-> \textbf{\emph{article}} per scrivere articoli (scientifici) senza capitoli
  \item<3-> \textbf{report} per scrivere relazioni o tesi suddivise in capitoli
  \item<4-> \textbf{letter} per scrivere lettere
  \item<5-> \textbf{\emph{book}} per scrivere libri
  \item<6-> {\footnotesize\textbf{memoir} per scrivere libri complessi}
  \item<7-> {\footnotesize\textbf{beamer} per presentazioni}
  \item<8-> \textbf{\dots}
  \end{itemize}
}

\begin{frame}[fragile]\transfade\centering
\frametitle{Esempio minimale (1/2)}
\begin{verbatim}
  \documentclass[a4paper, 12pt]{report}
    %    10pt, 11pt, 12 pt ↑      ↑ classe
    %    twocolumn dispone il testo su due colonne

  \pdfpagewidth\paperwidth      % per assicurarsi che nel pdf
  \pdfpageheight\paperheight    % sia riempita tutta la pagina

  \usepackage[italian]{babel}   % lingua usata nel documento
  \usepackage[utf8]{inputenc}  % se non va usare utf8x
  \usepackage[T1]{fontenc}
\end{verbatim}\dots\\~
\end{frame}
\begin{frame}[fragile]\transfade\centering
  \frametitle{Esempio minimale (2/2)}
  {\small~\\[-0.5cm]\dots\\[-0.5cm]\begin{verbatim}
    \begin{document}
      \title{Titolo del documento}\author{Autore}\date{Data}
      \maketitle

      \tableofcontents  %indice

      \chapter{Nome capitolo}
        \section{Nome sezione}
          \subsection{Nome sottosezione}
            \subsubsection{Nome sotto-sottosezione}
              testo

    \end{document}
  \end{verbatim}}
\end{frame}

\frame{\transfade\centering
\frametitle{Font}
\begin{itemize}
  \item<2->{\texttt{\textbf{\textbackslash textbf}}\{testo\} \textbf{grassetto}}
  \item<2->{\texttt{\textbf{\textbackslash emph}}\{testo\} \emph{italico} ("corsivo")}
  \item<2->{\texttt{\textbf{\textbackslash texttt}}\{testo\} \texttt{monospace}}
\end{itemize}
\begin{itemize}
  \item<3->{\{\texttt{\textbf{\textbackslash tiny}} testo \}}
  \item<3->{\{\texttt{\textbf{\textbackslash footnotesize}} testo \}}
  \item<3->{\{\texttt{\textbf{\textbackslash small}} testo \}}
  \item<3->{\{\texttt{\textbf{\textbackslash large}} testo \}}
  \item<3->{\{\texttt{\textbf{\textbackslash Large}} testo \}}
\end{itemize}
}

\frame{\transfade\centering
\frametitle{Ultimi dettagli}
\visible<2->{Andare a capo}
  \begin{itemize}
    \item<2->{\texttt{\textbf{\textbackslash\textbackslash}} a capo}
    \item<2->{\texttt{\textbf{\textbackslash par}} nuovo paragrafo}
  \end{itemize}
\visible<3->{Alcuni simboli}
  \begin{itemize}
    \item<3->{\texttt{\textbf{\textbackslash\$}} \$}
    \item<3->{\texttt{\textbf{\textbackslash\%}} \%}
    \item<3->{\texttt{\textbf{\textbackslash\_}} \_}
  \end{itemize}
}