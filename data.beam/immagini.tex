\section{Immagini}
\begin{frame}\transfade\centering
  \frametitle{Immagini in linea}
\visible<3->{\texttt{\textbackslash usepackage\{graphicx\}}}\\
\texttt{\textbackslash includegraphics\only<4->{[width=300px,height=50px]}\{disegno\}}\\~\\
\only<2-3>{\includegraphics[height=100px]{img/disegno}}
\only<4->{\includegraphics[width=300px,height=50px]{img/disegno}}~\\~
\visible<5->{PDF, PNG e JPEG}
\end{frame}
\begin{frame}\transfade\centering
  \frametitle{Perché in linea?}
\texttt{Questa è \textbackslash includegraphics\{disegno\} una prova}\\~\\
\only<2->{Questa è \includegraphics[height=100px]{img/disegno} una prova}
\only<3->{\\Ciao {\footnotesize😀}, come va?}
\end{frame}

\begin{frame}[fragile]\transfade\centering
  \frametitle{Immagini}
  {\small\verb!\begin{figure}                       !\\
  \verb!  \includegraphics{immagine}         !\\
  \verb!  \caption{Disegno astratto}         !\\
  \only<2->{\texttt{~~\textbackslash label\{fig:Disegno\}~~~~~~~~~~~~~~~~}\\}
  \verb!\end{figure}                         !}
  \begin{figure}[!ht]%
    \includegraphics[height=50px]{img/disegno}%
    \caption{Disegno astratto}%
    \label{fig:Disegno}%
  \end{figure}\vspace{-1.5em}%
  \only<2->{\small\texttt{Visibile in figura \textbackslash ref\{fig:Disegno\}} $\Rightarrow$ Visibile in figura \ref{fig:Disegno}\\~}
\end{frame}

