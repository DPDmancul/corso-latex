\section{Tipologie di documento}
\begin{itemize}
  \item \textbf{\emph{article}} per scrivere articoli (scientifici) senza capitoli
  \item \textbf{report} per scrivere relazioni o tesi suddivise in capitoli
  \item \textbf{letter} per scrivere lettere
  \item \textbf{\emph{book}} per scrivere libri
  \item {\footnotesize\textbf{memoir} per scrivere libri complessi}
  \item {\footnotesize\textbf{beamer} per presentazioni}
  \item \textbf{\dots}
\end{itemize}

\section{Esempio minimale}
\begin{verbatim}
  \documentclass[a4paper, twocolumn, 12pt]{report}
    %               10pt, 11pt, 12 pt ↑       ↑ classe
    %    twocolumn dispone il testo su due colonne

  \pdfpagewidth\paperwidth      % per assicurarsi che nel pdf
  \pdfpageheight\paperheight    % sia riempita tutta la pagina

  \usepackage[italian]{babel}   % lingua usata nel documento
  \usepackage[utf8]{inputenc}  % se non va usare utf8x o alla peggio latin1
  \usepackage[T1]{fontenc}

  \begin{document}
    \title{Titolo del documento}\author{Autore}\date{Data}
    \maketitle

    \tableofcontents  %indice

    \chapter{Nome capitolo}
      testo
      \section{Nome sezione}
        testo
        \subsection{Nome sottosezione}
          testo
          \subsubsection{Nome sotto-sottosezione}
            testo

  \end{document}
  \end{verbatim}

  \subsection{Andare a capo}
  \begin{itemize}
    \item{\texttt{\textbf{\textbackslash\textbackslash}} a capo}
    \item{\texttt{\textbf{\textbackslash par}} nuovo paragrafo}
  \end{itemize}
\subsection{Alcuni simboli}
  \begin{itemize}
    \item{\texttt{\textbf{\textbackslash\$}} \$}
    \item{\texttt{\textbf{\textbackslash\%}} \%}
    \item{\texttt{\textbf{\textbackslash\_}} \_}
  \end{itemize}
  In genere tutti i simboli che hanno un significato in \LaTeX{} (vedremo che avrenno questo ruolo \&, \$, \^{}, \_, \dots) si scrivono come sopra.

\section{Font}
  \subsection{Stili principali}
  \begin{itemize}
    \item{\texttt{\textbf{\textbackslash textbf}}\{testo\} \textbf{grassetto}}
    \item{\texttt{\textbf{\textbackslash emph}}\{testo\} \emph{italico} ("corsivo")}
    \item{\texttt{\textbf{\textbackslash texttt}}\{testo\} \texttt{monospace}}
  \end{itemize}
  \subsection{Dimensioni principali}
  \begin{itemize}
    \item{\{\texttt{\textbf{\textbackslash tiny}} \tiny testo \}}
    \item{\{\texttt{\textbf{\textbackslash footnotesize}} \footnotesize testo \}}
    \item{\{\texttt{\textbf{\textbackslash small}} \small testo \}}
    \item{\{\texttt{\textbf{\textbackslash large}} \large testo \}}
    \item{\{\texttt{\textbf{\textbackslash Large}} \Large testo \}}
  \end{itemize}