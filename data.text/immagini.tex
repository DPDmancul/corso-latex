\section{Immagini}
 \subsection{Immagini in linea}
  \texttt{\textbackslash usepackage\{graphicx\} \% da inserire prima dell'inizio del docuemnto}\\
  \texttt{\textbackslash includegraphics\{disegno\}}\\~\\
  \includegraphics[height=100px]{img/disegno}\\
  \texttt{\textbackslash includegraphics[width=300px,height=50px]\{disegno\}}\\~\\
  \includegraphics[width=300px,height=50px]{img/disegno}~\\~
  \texttt{\textbackslash includegraphics[width=\textbackslash textwidth]\{disegno\}}\\~\\
  \includegraphics[width=\textwidth]{img/disegno}~\\~
  Come argomento va inserito il nome del file senza estensione. Sono supportate immagini PDF, PNG e JPEG.\\
  Questo tipo di insierimento si dice in linea perché l'immagine "fa parte" del testo:\\
  \texttt{Questa è \textbackslash includegraphics\{disegno\} una prova}\\~\\
  Questa è \includegraphics[height=100px]{img/disegno} una prova\par
  Alcune librerie (come apple\_emoji) permettono di inserire emoji nel testo proprio sfruttando questo sistema.\\
  \verb!\usepackage{apple_emoji}!\\
  \texttt{Ciao \smiley, come va?} $\Rightarrow$ Ciao 😀, come va?
  \subsection{Immagini come figure}
  \verb!\begin{figure}                       !\\
	\\\verb!\centering %allinea l'immagine' al centro!\\
  \verb!  \includegraphics{immagine}         !\\
  \verb!  \caption{Disegno astratto}         !\\
  \verb!  \abel{fig:Disegno}                 !\\
  \verb!\end{figure}                         !\\
  \begin{figure}[!ht]\centering%
    \includegraphics[height=60px]{img/disegno}%
    \caption{Disegno astratto}%
    \label{fig:Disegno}%
  \end{figure}\\%
  \texttt{Visibile in figura \textbackslash ref\{fig:Disegno\}} $\Rightarrow$ Visibile in figura \ref{fig:Disegno}

