\section{Bibliografia}
\verb!\begin{thebibliography}{Numero massimo voci} !\\
  \verb!  \addcontentsline{toc}{section}{Bibliografia} !\\
  \verb!  \label{bibliografia} !\\
  ~\\
  \verb!  \bibitem{WiFiStandard} %etichetta!\\
  \verb!    Ian Poole,\\ !\\
  \verb!    \emph{IEEE 802.11 Wi-Fi Standards},\\ !\\
  \verb!    \url{http://www.radio-electronics.com/info/wireless/wi-fi/!\\
  \verb!         ieee-802-11-standards-tutorial.php},\\ !\\
  \verb!    consultato il 3 Giugno 2017. !\\
  ~\\
  \verb! \end{thebibliography} !\\~\par
  Il numero massimo di voci serve all'allineamento dei numeri, è consigliato scrivere tanti 8 quante le cifre del numero più grande (ad esempio se avete 15 libri in bibliografia scrivete 88).

      \begin{thebibliography}{8}
        \bibitem{WiFiStandard}
          Ian Poole,\\
          \emph{IEEE 802.11 Wi-Fi Standards},\\
          \url{http://www.radio-electronics.com/info/wireless/wi-fi/ieee-802-11-standards-tutorial.php},\\
          consultato il 3 Giugno 2017.
      \end{thebibliography}
\subsection{Richiamo della bibliografia}
\texttt{... che utilizza modulazioni digitali \textbackslash cite\{WiFiStandard\}}~\\~
... che utilizza modulazioni digitali~\cite{WiFiStandard}\\~


